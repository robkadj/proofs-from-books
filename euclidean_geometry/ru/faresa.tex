\documentclass{amsart}
\usepackage[utf8]{inputenc}
\usepackage[russian]{babel}
%\usepackage{faktor}
\usepackage{tkz-euclide}
\usetkzobj{all}
\begin{document}

\textbf{Т1. Обобщенная теорема Фалеса}

Параллельные прямые пересекающие стороны угла, отсекают от сторон угла пропорциональные отрезки

\textbf{Доказательство}

Пусть имеем угол K : $\angle K$, проведём три параллельных прямых на сторонах данного угла (рис. 1). Прямые расположены на разных расстояниях одной от другой. Прямые отсекают четыре отрезка на сторонах угла (смотрите рис. 1). Обозначим длинны этих отрезков следуюшими буквами $\overline{AB}$ = a, $\overline{BC}$ = b, $\overline{DE}$ = c и $\overline{EF}$ = d. Выполняются следующие неравенства : a > b и c > d.  По условию теоремы имеем $\frac{AB}{DE}$ = $\frac{BC}{EF}$ $\Leftrightarrow$ $\frac{a}{c}$ = $\frac{b}{d}$, делаем вывод : a $\cdot$ d = b $\cdot$ c.

Предположим что $\frac{a}{c}$ $\neq$ $\frac{b}{d}$, отсюда вытекает что a $\cdot$ d $\neq$ b $\cdot$ c, далее могут быть два случая :

(1) a $\cdot$ d > b $\cdot$ c

(2) a $\cdot$ d < b $\cdot$ c

(1) делим неравенство на a и получаем d > c $\cdot$ $\frac{b}{a}$, но по условию у нас c > d, что приводит к противоречию.

(2) делим неравенство на d и получаем a < b $\cdot$ $\frac{c}{d}$,  но по условию у нас a > b что приводит к противоречию.

В случае когда прямые расположены на одном и том же расстояний, используется та же стратегия, которая также приведет к противоречию.

Теорема доказана.



\end{document}
