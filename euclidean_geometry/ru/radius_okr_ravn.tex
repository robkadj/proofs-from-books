\documentclass{amsart}
\usepackage[utf8]{inputenc}
\usepackage[russian]{babel}
\usepackage{gensymb}
\usepackage{MnSymbol}
\usepackage{graphicx}
\graphicspath{ {./img/} }

\begin{document}
З1.Найдите радиус окружности, описанной около равнобедренного треугольника с основанием $a$ и с боковой стороной $b$ 

\textbf{Решение :}

Находим высоту $CD$ (смотрите рисунки a и b), опущенную на основание. По теореме Пифагора $CD = \sqrt{AC^2-AD^2} = \sqrt{b^2-\frac{a^2}{4}}$. Обозначим радиус описанной окружности через $R$. Если центр $O$ окружности лежит на высоте $CD$  (рис. a), то  $OD=CD-R$. Если центр окружности лежит на продолжение высоты (рис. b) то $OD=R-CD$. По теореме Пифагора $AO^2=OD^2+AD^2$, или \\*

$R^2=(\sqrt{b^2-\frac{a^2}{4}})-R)^2 + \frac{a^2}{4}

$\blacksquare$

\end{document}