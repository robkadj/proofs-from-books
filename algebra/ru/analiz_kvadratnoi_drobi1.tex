\documentclass{amsart}
\usepackage[utf8]{inputenc}
\usepackage[russian]{babel}
\usepackage{gensymb}
\usepackage{MnSymbol}
\usepackage{graphicx}

\begin{document}

\textbf{Выяснить, при каких значениях $x$ дробь $\frac{x^2+2x-3}{2x-x^2}$ положительна и при каких - отрицательна}

Знак дроби зависит от знака числителя и знаменателя (таблица 1)
\begin{center}
\begin{tabular}{ c c c c c }
 \hline
 дробь & + & - & - & +  \\ 
 числитель & - & - & + & +  \\  
 знаменатель & - & + & - & + \\ 
 \hline   
\end{tabular}
\end{center}

Решаем квадратное уравнение (числитель) $x^2+2x-3=0$ \\
$D = b^2 - 4ac = 4 + 4 * 1 * 3 = 16$ \\
$\sqrt{D} = 4$ \\
$x_1 = \frac{b + \sqrt{D}}{2a} = \frac{-2 + 4}{2} = \frac{2}{2} = 1$ \\
$x_2 = \frac{b - \sqrt{D}}{2a} = \frac{-2 - 4}{2} = \frac{-6}{2} = -3$ \\

Решаем квадратное уравнение (знаменатель) $-x^2 + 2x + 0 = 0$ \\
$D = b^2 - 4ac = 4 + 4 * 0 = 4$ \\
$\sqrt{D} = 2$ \\
$x_1 = \frac{b + \sqrt{D}}{2a} = \frac{-2 + 2}{-2} = 0$ \\
$x_2 = \frac{b - \sqrt{D}}{2a} = \frac{-2 - 2}{-2} = \frac{-4}{-2} = 2$ \\

Из вышеприведенных результатах можем сделать следующие выводы : \\

Знак числителя (таблица 2) - \\
\begin{center}
 \begin{tabular}{c c}
 \hline
     Значение x & Знак \\
     $x < -3$ and $x > 1$ & $ + $ \\
     $-3 < x < 1$ & $ - $ \\ 
 \hline
 \end{tabular}
\end{center}

Знак знаменателя (таблица 3) - \\
\begin{center}
 \begin{tabular}{c c}
 \hline
     Значение x & Знак \\
     $0 < x < 2$ & $ + $ \\
     $x < 0$ and $x > 2$ & $ - $ \\ 
 \hline    
 \end{tabular}
\end{center}

Комбинируя данные из таблиц 1, 2 и 3 получаем :

\begin{center}
 \begin{tabular}{c c c c c}
 \hline
     дробь         &      +              &       -          &        -                 &      +                 \\
	 числитель     & $-3 < x < 1$        &    $-3 < x < 1$  &   $x < -3$ or $x > 1$   &  $x < -3$ or $x > 1$  \\
	 знаменатель   & $x < 0$ or $x > 2$ &    $0 < x < 2$   &   $x < 0$ or $x > 2$    &  $0 < x < 2$            \\
	 пересечение   & $-3 < x < 0$        &    $0 < x < 1 $  &   $x < -3 $ or $x > 2$   &  $\emptyset$            \\
  \hline
 \end{tabular}
\end{center}

Надо отметить что дробь превращается в 0 (то есть не имеет никакого знака) при $x_1 = 1$ и $x_2 = -3$, так как сам числитель превращается в 0. И не емеет значения при $x_1 = 0$ и $x_2 = 2$, так как знаменатель 0.

$\blacksquare$
\end{document}