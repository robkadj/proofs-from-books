\documentclass{amsart}
\usepackage[utf8]{inputenc}
\usepackage[russian]{babel}
\usepackage{gensymb}
\usepackage{MnSymbol}
\usepackage{graphicx}

\begin{document}

\textbf{Как строится график квадратной функций $y=ax^2+bx+c=0$ ?} Наше рассуждение разделено на несколько этапов :\\*

1) \textbf{График функции $y=x^2$} \\*
Этот график строится по точкам. График представляет собой параболу с вершиной в точке $(0,0)$, кривые параболы направлены вверх, и вся парабола находится над осью абсцисс. Функция $y=x^2$ принимает только неотрицательные значения. Также надо добавить что парабола симметрична относительно оси ординат. \\*

2) \textbf{График функции $y=\alpha x^2$ } \\*
Если $\alpha > 0$ то кривые параболы направлены вверх, если $\alpha < 0$, то кривые параболы направлены в низ. Вершиной параболы так как и в случае (1) является точка $(0,0)$, чем больше абсолютная величина $\alpha$, тем круче ветви параболы. \\*

3) \textbf{График функций $y = \alpha (x - \beta)^2$} \\*
Вся кривая $y = \alpha (x - \beta)^2$ получается посредством переноса кривой $y=\alpha x^2$ в право на $\beta$. Вершиной параболы $y = \alpha (x - \beta)^2$ будет точка с координатами $(\beta,0)$, а осью симметрии прямая $x=\beta$. Точьно так же может быть построен и график функций $y = \alpha (x + \beta)^2$, где $\beta > 0$. Он представляет собой параболу, которая получается посредством смещения параболы $y=x^2$ в лево на $x=\beta$. Вершина ее находится в точке с координатами $(-\beta,0)$; осью симметрий служит прямая $x=-\beta$ \\*

4) \textbf{График функций $y = \alpha (x - \beta)^2 + \gamma$} \\*
Этот график получается посредством смешения параболы $y = \alpha (x - \beta)^2$ по направлению оси $y$ вверх на расстояние $\gamma$, если $\gamma > 0$, и в низ на расстояние $-\gamma$, если $\gamma < 0$. В результате смешения получается парабола с вершиной в точке $(\beta,\gamma)$. Осью симметрий такой параболы служит прямая $x = \beta$. \\*

5) \textbf{График функций $y = ax^2+bx+c$} \\*
Квадратный трехчлен $ax^2+bx+c$ представим в виде $ax^2+bx+c = a(x+\frac{b}{2a})^2 - \frac{b^2 - 4ac}{4a}$. Последнее выражение имеет вид $\alpha (x - \beta)^2 + \gamma$. Поэтому графиком функций $y = ax^2+bx+c$ является парабола с вершиной в точке $(-\frac{b}{2a}, -\frac{b^2-4ac}{4a})$. Осью симметрий этой параболы является прямая $x = -\frac{b}{2a}$. При $a>0$, ветви параболы направлены вверх, а при $a<0$ - вниз.

$\blacksquare$

\end{document}